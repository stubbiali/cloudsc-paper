\documentclass[main.tex]{subfiles}

\begin{document}
    \justifying

    \section{Defining the targeted scientific applications}
    \label{section:target-cloud-microphysics-schemes}

        Several physical and chemical mechanisms occurring in the atmosphere are active on spatial scales that are significantly smaller than the highest affordable model resolution. It follows that these mechanisms cannot be properly captured by the resolved model dynamics, but need to be \emph{parametrized}. Parametrizations express the bulk effect of subgrid-scale phenomena on the resolved flow in terms of the grid-scale variables. The equations underneath physical parametrizations are based on theoretical and semi-empirical arguments, and their numerical treatment commonly adheres to the \emph{single-column} abstraction, so that adjustments can only happen within individual columns, with no data dependencies between columns. The atmospheric module of the IFS includes parametrizations dealing with the radiative heat transfer, deep and shallow convection, clouds and stratiform precipitation, surface exchange, turbulent mixing in the planetary boundary layer, subgrid-scale orographic drag, non-orographic gravity wave drag, and methane oxidation \citep{ifs48r1}.

        The focus of this paper is on the cloud microphysics modules of the ECMWF: the CLOUDSC -- used in operational forecasting -- and the CLOUDSC2 -- employed in the data assimilation. The motivation is three-fold: (i) both schemes are among the most computationally expensive parametrizations, with the CLOUDSC accounting for up to $10\%$ of the total execution time of the high-resolution operational forecasts at ECMWF; (ii) they are representative of the computational patterns ubiquitous in physical parametrizations; and (iii) they already exist in the form of \emph{dwarfs}. The weather and climate ``computational dwarfs'', or simply ``dwarfs'', are model components shaped into stand-alone software packages to serve as archetypes of relevant computational motifs \citep{muller19} and provide a convenient platform for performance optimizations and portability studies \citep{bauer20}. In recent years, the Performance and Portability Team of ECMWF created the CLOUDSC and CLOUDSC2 dwarfs. The original Fortran codes for both packages, corresponding respectively to the IFS Cycle 41r2 and 46r1, have been pulled out of the IFS codebase, slightly polished\footnote{\review{Compared to the original implementations run operationally at ECMWF, the CLOUDSC \& CLOUDSC2 dwarf codes do not include (i) all the IFS-specific infrastructure code, (ii) the calculation of budget diagnostics, and (iii) dead code paths.}} and finally made available in public code repositories\footnote{\url{https://github.com/ecmwf-ifs/dwarf-p-cloudsc} and \url{https://github.com/ecmwf-ifs/dwarf-p-cloudsc2-tl-ad}}. Later, the repositories have been enriched with alternative coding implementations, using different languages and programming paradigms; the most relevant implementations will be discussed in Section \ref{section:performance-analysis}.

    \subsection{CLOUDSC: Cloud microphysics of the forecast model}

        The CLOUDSC is a single-moment cloud microphysics scheme that parametrizes stratiform clouds and their contribution to surface precipitation \citep{ifs48r1}. It was implemented in the IFS Cycle 36r4 and has been operational at ECMWF since November 2010. Compared to the pre-existing scheme, it accounts for five prognostic variables (cloud fraction, cloud liquid water, cloud ice, rain and snow) and brings substantial enhancements in different aspects, including treatment of mixed-phase clouds, advection of precipitating hydrometeors (rain and snow), physical realism, and numerical stability \citep{nogherotto16}. For a comprehensive description of the scheme, we refer the reader to \citet{forbes11b} and the references therein. For all the coding versions considered in this paper, including the novel Python rewrite, the calculations are validated by direct comparison of the output against serialized language-agnostic reference data provided by ECMWF.

    \subsection{CLOUDSC2: Cloud microphysics in the context of data assimilation}
    \label{section:cloudsc2}

        The CLOUDSC2 scheme represents a streamlined version of CLOUDSC, devised for use in the four-dimensional variational assimilation (4D-Var) at ECMWF \citep{courtier94}. 4D-Var merges short-term model integrations with observations over a twelve-hour assimilation window to determine the best possible representation of the current state of the atmosphere. This then provides the initial conditions for longer-term forecasts \citep{janiskova23}. The optimal synthesis between model and observational data is found by minimizing a cost function, which is evaluated using the \emph{tangent-linear} of the \emph{non-linear} forecasting model, while the \emph{adjoint} model is employed to compute the gradient of the cost function \citep{errico97, janiskova99}. For the sake of computational economy, the tangent-linear and adjoint operators are derived from a simplified and regularized version of the full non-linear model. The CLOUDSC2 is one of the physical parametrizations included in the ECMWF's simplified model, together with radiation, vertical diffusion, orographic wave drag, moist convection, and non-orographic gravity wave activity \citep{janiskova23}. In the following, we provide a mathematical and algorithmic representation of the tangent-linear and adjoint versions of CLOUDSC2. For the sake of brevity, in the rest of the paper we will refer to the non-linear, tangent-linear and adjoint formulations of CLOUDSC2 using CLOUDSC2NL, CLOUDSC2TL and CLOUDSC2AD, respectively.

        Let $F: \boldsymbol{x} \mapsto \boldsymbol{y}$ be the functional description of CLOUDSC2, connecting the input fields $\boldsymbol{x}$ with the output variables $\boldsymbol{y}$. The tangent-linear operator $F'$ of $F$ is derived from the Taylor series expansion
        \begin{equation}
            F \left( \boldsymbol{x} + \delta \boldsymbol{x} \right) = \boldsymbol{y} + \delta \boldsymbol{y} = F \left( \boldsymbol{x} \right) + F' \left[ \boldsymbol{x} \right] \left( \delta \boldsymbol{x} \right) + \mathcal{O} \left( ||\delta \boldsymbol{x} ||^2 \right) \, ,
        \end{equation}
        where $\delta \boldsymbol{x}$ and $\delta \boldsymbol{y}$ are variations on $\boldsymbol{x}$ and $\boldsymbol{y}$, and $|| \cdot ||$ is a suitable norm. The formal correctness of the coding implementation of $F'$ can be assessed through the Taylor test (also called the ``V-shape'' test), which ensures that the following condition is satisfied up to machine precision:
        \begin{equation}
            \lim_{\lambda \to 0} \dfrac{F \left( x + \lambda \delta \boldsymbol{x} \right) - F \left( \boldsymbol{x} \right)}{F' \left[ \boldsymbol{x} \right] \left( \lambda \delta \boldsymbol{x} \right)} = 1 \qquad \forall \, \boldsymbol{x}, \, \delta \boldsymbol{x} \, .
        \end{equation}
        The logical steps carried out in the actual implementation of the Taylor test are sketched in Algorithm \ref{alg:taylor-test} \review{(Appendix \ref{section:appendix})}.

        The adjoint operator $F^*$ of $F'$ is defined such that for the inner product $< \cdot, \, \cdot >$:
        \begin{equation}
            \label{eq:adjoint}
            < \delta \boldsymbol{x}, \, F^* \left[ \boldsymbol{y} \right] \left( \delta \boldsymbol{y} \right) > ~ = ~ < \delta \boldsymbol{y}, \, F' \left[ \boldsymbol{x} \right] \left( \delta \boldsymbol{x} \right) > \qquad \forall \, \boldsymbol{x}, \, \delta \boldsymbol{x}, \, \boldsymbol{y}, \, \delta \boldsymbol{y} \, .
        \end{equation}
        In particular, \eqref{eq:adjoint} must hold for $\boldsymbol{y} = F \left( \boldsymbol{x} \right)$ and $\delta \boldsymbol{y} = F' \left[ \boldsymbol{x} \right] \left( \delta \boldsymbol{x} \right)$:
        \begin{equation}
            \label{eq:symmetry-test}
            < \review{\delta} \boldsymbol{x}, \, F^* \left[ F \left( \boldsymbol{x} \right) \right] \left( F' \left[ \boldsymbol{x} \right] \left( \delta \boldsymbol{x} \right) \right) > ~ = ~ < F' \left[ \boldsymbol{x} \right] \left( \delta \boldsymbol{x} \right), \, F' \left[ \boldsymbol{x} \right] \left( \delta \boldsymbol{x} \right) > \qquad \forall \, \boldsymbol{x}, \, \delta \boldsymbol{x} \, .
        \end{equation}
        The latter condition is at the \review{heart} of the so-called symmetry test for $F^*$ (see Algorithm \ref{alg:symmetry-test} \review{in Appendix \ref{section:appendix}}).

    %\biblio
\end{document}

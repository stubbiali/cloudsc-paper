\documentclass[gmd,manuscript,online]{copernicus}

% ============================================================
% packages: base packages and hardly taggable stuff
% ============================================================
\usepackage[utf8]{inputenc}
\usepackage[T1]{fontenc}
\usepackage{newunicodechar}
\usepackage{lmodern}
\usepackage{anyfontsize}
\usepackage{enumitem}  % in conflict with \usepackage{enumerate}
%\usepackage{fourier}  % style
%\usepackage{lipsum}  % dummy text
%\usepackage{authblk}  % footnote style author/affiliation
%\usepackage[top=2cm, bottom=3cm, left=2cm, right=2cm, scale=0.75]{geometry}  % set the margins
%\usepackage[inner=3cm, textwidth=445pt, scale=0.75, top=3cm, bottom=3cm]{geometry}  % set the margins (book style)
\usepackage{fancyvrb}  % insert verbatim within a command
\usepackage{fancyhdr}  % set header and footer
\usepackage[letterspace=150]{microtype}  % improved typographics
\usepackage{textcomp}  % more text symbols
% \usepackage{gensymb}  % more text and math symbols - clash!!!
\usepackage{bbm}  % blackboard-style cm fonts
\usepackage{csvsimple}  % csv file processing
\usepackage[bottom]{footmisc}  % footnotes at the bottom of the page
\usepackage{stackengine}  % stack objects vertically in a variety of customizable ways
\usepackage{ragged2e}  % \justify

% new page
\newcommand{\NewPage}{\newpage\null\thispagestyle{empty}\newpage}

% no indent
%\setlength\parindent{0pt}

% ============================================================
% packages: math
% ============================================================
\usepackage{bm}  % access bold symbols in math mode
\usepackage{amsmath,etoolbox}
\allowdisplaybreaks
\usepackage{amsfonts}
\usepackage{amstext}
\usepackage{amssymb}
\usepackage{amsthm}
\usepackage{empheq}
\usepackage{cases}
\usepackage{mathtools}
\usepackage{nicefrac}  % diagonal fractions

% ============================================================
% math aliases and commands
% ============================================================
% enumerate subequations with arabic numbers (e.g. 1.1, 1.2, ecc)
\patchcmd{\subequations}{\def\theequation{\theparentequation\alph{equation}}}{\def\theequation{\theparentequation.\arabic{equation}}}{}{}

%\usepackage{chngcntr}
%\counterwithout{equation}{section} % undo numbering system provided by phstyle.cls
%\counterwithin{equation}{chapter}  % implement desired numbering system

% absolute value
\DeclarePairedDelimiter\abs{\lvert}{\rvert}
\makeatletter
\let\oldabs\abs
\def\abs{\@ifstar{\oldabs}{\oldabs*}}

% argmin
\DeclareMathOperator*{\argmin}{arg\,min}

% norm symbol
\newcommand{\norm}[1]{\left\lVert#1\right\rVert}

% mod symbol
\newcommand{\Mod}[1]{\ (\mathrm{mod}\ #1)}

% aliases for \widetilde
\newcommand{\wt}[1]{\widetilde{#1}}

% make \big| adapt to the context
\makeatletter
\let\amstexbig\big
\def\newbig#1{%
  \ifx#1|%
	\expandafter\@firstoftwo
  \else
	\expandafter\@secondoftwo
  \fi
  {\big@bar}%
  {\amstexbig{#1}}%
}
\AtBeginDocument{\let\big\newbig}
\def\big@bar{\bBigg@{1.1}|}
\makeatother

% theorem and definition environment
\theoremstyle{theorem}
\newtheorem{theorem}{Theorem}[section]
\theoremstyle{definition}
\newtheorem{definition}{Definition}[section]
\theoremstyle{remark}
\newtheorem{remark}{Remark}[section]
\theoremstyle{proposition}
\newtheorem{proposition}{Proposition}[section]
%\newenvironment{definition}[1][Definition]{\begin{trivlist}
%\item[\hskip \labelsep {\bfseries #1}]}{\end{trivlist}}

% enumerate the equations and the figures according to the section they are in
%\numberwithin{equation}{section}
%\numberwithin{figure}{chapter}

% inline fractions settings
\renewcommand{\textfraction}{0.1}
\renewcommand{\topfraction}{0.9}

% matrices
\newcommand{\matr}[1]{\mathbf{#1}}  % undergraduate algebra version
%\newcommand{\matr}[1]{#1}          % pure math version
%\newcommand{\matr}[1]{\bm{#1}}     % ISO complying version

% interrupting and resuming subequations
\makeatletter
\newcounter{qrr@oldeq}
\newcounter{qrr@oldsubeq}
\newcounter{qrr@realeq}
\renewenvironment{subequations}{%
  \refstepcounter{equation}%
  \protected@edef\theparentequation{\theequation}%
  \setcounter{parentequation}{\value{equation}}%
  \setcounter{equation}{0}%
  \def\theequation{\theparentequation\alph{equation}}%
  \ignorespaces
}{%
  \setcounter{qrr@oldeq}{\value{parentequation}}%
  \setcounter{qrr@oldsubeq}{\value{equation}}%
  \setcounter{equation}{\value{parentequation}}%
  \ignorespacesafterend
}
\newenvironment{subequations*}{%
  \setcounter{qrr@realeq}{\value{equation}}%
  \let\theparentequation\theequation%
  \patchcmd{\theparentequation}{equation}{parentequation}{}{}%
  \setcounter{parentequation}{\numexpr\value{qrr@oldeq}-1}%
  \setcounter{equation}{\value{qrr@oldsubeq}}%
  \def\theequation{\theparentequation\alph{equation}}%
  \refstepcounter{parentequation}%
  \ignorespaces
}{%
  \setcounter{qrr@oldeq}{\value{parentequation}}%
  \setcounter{qrr@oldsubeq}{\value{equation}}%
  \setcounter{equation}{\value{qrr@realeq}}%
  \ignorespacesafterend
}
\makeatother

% aliases for greek letters
\newcommand{\bchi}{\bm{\chi}}
\newcommand{\bphi}{\bm{\phi}}
\newcommand{\bpsi}{\bm{\psi}}
\newcommand{\bxi}{\bm{\xi}}

% imaginary unit
\newcommand{\iu}{\text{i}}

% math environments patched to enable proper line numbering
% \newenvironment{myeq}{\begin{linenomath*}\begin{equation}}{\end{equation}\end{linenomath*}}
% \newenvironment{myeq*}{\begin{linenomath*}\begin{equation*}}{\end{equation*}\end{linenomath*}}
% \newenvironment{myalign}{\begin{linenomath*}\begin{align}}{\end{align}\end{linenomath*}}
% \newenvironment{myalign*}{\begin{linenomath*}\begin{align*}}{\end{align*}\end{linenomath*}}

% ============================================================
% packages: tabular
% ============================================================
\usepackage{longtable}
\usepackage{tabu}
\usepackage{multicol}
\usepackage{multirow}
\usepackage{array}
\usepackage{makecell}
\usepackage{booktabs}
\usepackage{hhline}

% ============================================================
% packages: hyperlinks
% ============================================================
% \RequirePackage[hyphens]{url}
% \PassOptionsToPackage{hyphens}{url}\usepackage{hyperref}
\usepackage{breakurl}
\usepackage{url}

% ============================================================
% packages: graphics
% ============================================================
%\usepackage{floatrow}	% notes below a figure
\usepackage{pdfpages}  % include pdf documents
\usepackage{graphicx}  	
\usepackage{epstopdf}
%\epstopdfsetup{outdir=/Users/subbiali/Desktop/phd/manuscripts/pdc_paper/img/epstopdf/}
\usepackage{float}
\usepackage[position = top]{subfig}
\usepackage{wrapfig}
\usepackage[labelfont=bf,labelsep=period,font=small]{caption}
% \usepackage{subcaption}  % in conflict with \usepackage{subfloat}
\usepackage{tikz}
\usetikzlibrary{decorations.pathreplacing}

% modify the space between figure and caption
%\setlength{\abovecaptionskip}{-4pt}
%\setlength{\belowcaptionskip}{3pt}

% ============================================================
% packages: pseudocode
% ============================================================
\usepackage{algorithm}
\usepackage[noend]{algpseudocode}

% fancy "for" block
\algnewcommand\algorithmicto{\textbf{to}}
\algrenewtext{For}[3]%
{\algorithmicfor\ $#1 \gets #2$ \algorithmicto\ $#3$ \algorithmicdo}

% redefine \Require and \Ensure for algorithm environment
%\renewcommand{\algorithmicrequire}{\textbf{Input:}}
%\renewcommand{\algorithmicensure}{\textbf{Output:}}

% define the do-while loop
%\algdef{SE}[DOWHILE]{DoWhile}{EndDoWhile}{\algorithmicdo}[1]{\algorithmicwhile\ #1}

% ============================================================
% packages: fancy boxes
% ============================================================
\usepackage[export]{adjustbox}

\usepackage[many]{tcolorbox}
%\makeatletter
%	\renewcommand*\l@figure{\@dottedtocline{1}{1em}{3.2em}}
%\makeatother

% ============================================================
% packages: listings
% ============================================================
\usepackage{listings}

% minted: package for formatting and highlighting source code
\usepackage[cache=true]{minted}
\usemintedstyle{default}
\setminted{mathescape, escapeinside=||, tabsize=4, fontfamily=tt}
\newmint[py]{python}{python3=true}
\newmintinline[pyinline]{python}{python3=true}

% ============================================================
% packages: bibliography
% ============================================================
% \usepackage[authoryear, round, sort, nonamebreak]{natbib}
\bibliographystyle{copernicus}

% ============================================================
% graphics path
% ============================================================
\graphicspath{{./../../img}}

% ============================================================
% packages: template specific
% ============================================================
\usepackage{lineno}
% \usepackage[inline]{trackchanges} %for better track changes. finalnew option will compile document with changes incorporated.
\usepackage{soul}
\linenumbers

% ============================================================
% subfiles
% ============================================================
\usepackage{subfiles}

% allow \cite in subfiles
\makeatletter
\patchcmd{\nocite}{\ifx\@onlypreamble\document}{\iftrue}{}{}
\makeatother

% make references pop-up in subfiles
% note: this does not seem to work...
\def\biblio{\bibliography{references}}

% ============================================================
% line spacing
% ============================================================
\usepackage{setspace}
%\doublespacing

% ============================================================
% begin document
% ============================================================
\begin{document}
	\justify

	\def\biblio{}

	% ============================================================
	% title, authors and affiliations
	% ============================================================
	\subfile{subfiles/title.tex}
	\doublespacing

	% ============================================================
	% abstract
	% ============================================================
	\subfile{subfiles/abstract.tex}

	% ============================================================
	% sections
	% ============================================================
    \subfile{subfiles/section01.tex}
	\subfile{subfiles/section02.tex}
	\subfile{subfiles/section03.tex}
	\subfile{subfiles/section04.tex}
	\subfile{subfiles/section05.tex}
	\subfile{subfiles/section06.tex}

	% ============================================================
	% copyright
	% ============================================================
	%\copyrightstatement{TODO}

	% ============================================================
	% code and data availability
	% ============================================================
	%% use this section when having only software code available
	%\codeavailability{TEXT}

	%% use this section when having only data sets available
	%\dataavailability{TEXT}

	%% use this section when having data sets and software code available
	%\codedataavailability{TODO}

	%% use this section when having geoscientific samples available
	%\sampleavailability{TEXT}

	%% use this section when having video supplements available
	%\videosupplement{TEXT}

	% ============================================================
	% appendix (optional)
	% ============================================================
	% The \appendix command resets counters and redefines section heads
	%
	% After typing \appendix
	%
	%\section{Here Is Appendix Title}
	% will show
	% A: Here Is Appendix Title
	%
	% \noappendix marks the end of the appendix section.
	%
	% Regarding figures and tables in appendices, the following two options are possible depending on your general handling of figures and tables in the manuscript environment:
	%
	% Option 1: If you sorted all figures and tables into the sections of the text, please also sort the appendix figures and appendix tables into the respective appendix sections.
	% They will be correctly named automatically.
	%
	% Option 2: If you put all figures after the reference list, please insert appendix tables and figures after the normal tables and figures.
	% To rename them correctly to A1, A2, etc., please add the following commands in front of them:
	%
	% \appendixfigures  %% needs to be added in front of appendix figures
	%
	% \appendixtables   %% needs to be added in front of appendix tables
	%
	% Please add \clearpage between each table and/or figure. Further guidelines on figures and tables can be found below.
	\appendix
	\subfile{subfiles/appendix01.tex}

	% ============================================================
	% glossary, notation and acronyms (optional)
	% ============================================================
	% Glossary is only allowed in Reviews of Geophysics
	%  \begin{glossary}
	%  \term{Term}
	%   Term Definition here
	%  \term{Term}
	%   Term Definition here
	%  \term{Term}
	%   Term Definition here
	%  \end{glossary}
	%
	% Acronyms
	%   \begin{acronyms}
	%   \acro{Acronym}
	%   Definition here
	%   \acro{EMOS}
	%   Ensemble model output statistics
	%   \acro{ECMWF}
	%   Centre for Medium-Range Weather Forecasts
	%   \end{acronyms}
	%
	% Notation
	%   \begin{notation}
	%   \notation{$a+b$} Notation Definition here
	%   \notation{$e=mc^2$}
	%   Equation in German-born physicist Albert Einstein's theory of special
	%  relativity that showed that the increased relativistic mass ($m$) of a
	%  body comes from the energy of motion of the body—that is, its kinetic
	%  energy ($E$)—divided by the speed of light squared ($c^2$).
	%   \end{notation}

	% ============================================================
	% acknowledgments
	% ============================================================
	\subfile{subfiles/acknowledgements.tex}

	% ============================================================
	% bibliography
	% ============================================================
	%% Since the Copernicus LaTeX package includes the BibTeX style file copernicus.bst,
	%% authors experienced with BibTeX only have to include the following two lines:
	%%
	%% \bibliographystyle{copernicus}
	%% \bibliography{example.bib}
	%%
	%% URLs and DOIs can be entered in your BibTeX file as:
	%%
	%% URL = {http://www.xyz.org/~jones/idx_g.htm}
	%% DOI = {10.5194/xyz}


	%% LITERATURE CITATIONS
	%%
	%% command                        & example result
	%% \citet{jones90}|               & Jones et al. (1990)
	%% \citep{jones90}|               & (Jones et al., 1990)
	%% \citep{jones90,jones93}|       & (Jones et al., 1990, 1993)
	%% \citep[p.~32]{jones90}|        & (Jones et al., 1990, p.~32)
	%% \citep[e.g.,][]{jones90}|      & (e.g., Jones et al., 1990)
	%% \citep[e.g.,][p.~32]{jones90}| & (e.g., Jones et al., 1990, p.~32)
	%% \citeauthor{jones90}|          & Jones et al.
	%% \citeyear{jones90}|            & 1990
	\bibliography{subfiles/references}

% ============================================================
% end document
% ============================================================
\end{document}

% ============================================================
% more information and advice
% ============================================================
%% FIGURES

%% When figures and tables are placed at the end of the MS (article in one-column style), please add \clearpage
%% between bibliography and first table and/or figure as well as between each table and/or figure.

% The figure files should be labelled correctly with Arabic numerals (e.g. fig01.jpg, fig02.png).


%% ONE-COLUMN FIGURES

%%f
%\begin{figure}[t]
%\includegraphics[width=8.3cm]{FILE NAME}
%\caption{TEXT}
%\end{figure}
%
%%% TWO-COLUMN FIGURES
%
%%f
%\begin{figure*}[t]
%\includegraphics[width=12cm]{FILE NAME}
%\caption{TEXT}
%\end{figure*}
%
%
%%% TABLES
%%%
%%% The different columns must be seperated with a & command and should
%%% end with \\ to identify the column brake.
%
%%% ONE-COLUMN TABLE
%
%%t
%\begin{table}[t]
%\caption{TEXT}
%\begin{tabular}{column = lcr}
%\tophline
%
%\middlehline
%
%\bottomhline
%\end{tabular}
%\belowtable{} % Table Footnotes
%\end{table}
%
%%% TWO-COLUMN TABLE
%
%%t
%\begin{table*}[t]
%\caption{TEXT}
%\begin{tabular}{column = lcr}
%\tophline
%
%\middlehline
%
%\bottomhline
%\end{tabular}
%\belowtable{} % Table Footnotes
%\end{table*}
%
%%% LANDSCAPE TABLE
%
%%t
%\begin{sidewaystable*}[t]
%\caption{TEXT}
%\begin{tabular}{column = lcr}
%\tophline
%
%\middlehline
%
%\bottomhline
%\end{tabular}
%\belowtable{} % Table Footnotes
%\end{sidewaystable*}
%
%
%%% MATHEMATICAL EXPRESSIONS
%
%%% All papers typeset by Copernicus Publications follow the math typesetting regulations
%%% given by the IUPAC Green Book (IUPAC: Quantities, Units and Symbols in Physical Chemistry,
%%% 2nd Edn., Blackwell Science, available at: http://old.iupac.org/publications/books/gbook/green_book_2ed.pdf, 1993).
%%%
%%% Physical quantities/variables are typeset in italic font (t for time, T for Temperature)
%%% Indices which are not defined are typeset in italic font (x, y, z, a, b, c)
%%% Items/objects which are defined are typeset in roman font (Car A, Car B)
%%% Descriptions/specifications which are defined by itself are typeset in roman font (abs, rel, ref, tot, net, ice)
%%% Abbreviations from 2 letters are typeset in roman font (RH, LAI)
%%% Vectors are identified in bold italic font using \vec{x}
%%% Matrices are identified in bold roman font
%%% Multiplication signs are typeset using the LaTeX commands \times (for vector products, grids, and exponential notations) or \cdot
%%% The character * should not be applied as mutliplication sign
%
%
%%% EQUATIONS
%
%%% Single-row equation
%
%\begin{equation}
%
%\end{equation}
%
%%% Multiline equation
%
%\begin{align}
%& 3 + 5 = 8\\
%& 3 + 5 = 8\\
%& 3 + 5 = 8
%\end{align}
%
%
%%% MATRICES
%
%\begin{matrix}
%x & y & z\\
%x & y & z\\
%x & y & z\\
%\end{matrix}
%
%
%%% ALGORITHM
%
%\begin{algorithm}
%\caption{...}
%\label{a1}
%\begin{algorithmic}
%...
%\end{algorithmic}
%\end{algorithm}
%
%
%%% CHEMICAL FORMULAS AND REACTIONS
%
%%% For formulas embedded in the text, please use \chem{}
%
%%% The reaction environment creates labels including the letter R, i.e. (R1), (R2), etc.
%
%\begin{reaction}
%%% \rightarrow should be used for normal (one-way) chemical reactions
%%% \rightleftharpoons should be used for equilibria
%%% \leftrightarrow should be used for resonance structures
%\end{reaction}
%
%
%%% PHYSICAL UNITS
%%%
%%% Please use \unit{} and apply the exponential notation

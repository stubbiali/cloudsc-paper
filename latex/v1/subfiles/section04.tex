\documentclass[main.tex]{subfiles}

\begin{document}
    \justifying

    \section{Infrastructure code}
    \label{section:infrastructure-code}

        % There exist several examples of parametrizations that have been devised for a specific model, and that can be transferred to another model only upon a significant amount of work and complex interfaces \citep{randall96}. Such an approach entails a lot of duplicated code and hinders the transfer of knowledge between research groups and modeling systems. This is in direct conflict, for instance, with the need of a comprehensive assessment of the impact of the time-stepping on weather forecasts and climate projections \citep{ubbiali21}. The recognition of the need for standardizing Earth system models dates back to the 1980s \citep{pielke84}. In \cite{kalnay89}, the authors suggested a list of basic programming rules to design \emph{plug-compatible} physics packages, which enable a high degree of scientific code exchange. This led to the idea of a common software infrastructure that couples different components while enhancing inter-operability, usability, software reuse, and portability \citep{dickinson02}. Since then, several frameworks have appeared that comply with this view \citep[][and references therein]{theurich16}, all identifying components with the major physical domains that constitute the Earth system: atmosphere, ocean, land surface, sea ice, ice sheet, and biogeochemistry. Recently, the Common Community Physics Package initiative \citep[CCPP;][]{heinzeller19} has lowered the concept of components to the level of individual physical processes. This design choice gives the ability to choose the order of parametrizations, to subcycle individual parametrizations, and to interleave dynamics with physics computations.

        % The same approach fostered by CCPP has been pursued by Sympl \citep{monteiro18}, a toolset of Python abstract base classes (ABCs) and utilities to write self-contained, self-documented and inter-changeable model components. Components interact through dictionaries whose keys are the names of the model variables (fields), and whose values are xarray's \pyinline{DataArray}s \citep{hoyer17} collecting the grid point values, the labelled dimensions, the axis coordinates, and the units for those variables. The most relevant component exposed by Sympl is \pyinline{TendencyComponent}, producing tendencies for prognostic variables and retrieving diagnostics. The class defines a minimal interface to declare the list of input and output fields, and initialize and run an instance of the class. This imposes minor constraints on model developers when writing a new physics package.

        All stencils of CLOUDSC-GT4Py and CLOUDSC2-GT4Py are defined, compiled and invoked within classes that leverage the functionalities provided by the Sympl package \citep{monteiro18}. Sympl is a toolset of Python utilities to write self-contained and self-documented model components. Because the components share a common Application Public Interface (API), they favor modularity, composability and inter-operability \citep{schaer19}. These aspects are of utter importance, for instance, when it comes to assessing the impact of process coupling on weather forecasts and climate projections \citep{ubbiali21}.

        \begin{listing}[t!]
            \begin{minted}{py}
import cupy as cp
from functools import cached_property
import numpy as np
from typing import Optional, Union
from ifs_physics_common.framework.components import DiagnosticComponent
from ifs_physics_common.framework.config import GT4PyConfig
from ifs_physics_common.framework.grid import ComputationalGrid, I, J, K

# type alias originally defined in ifs_physics_common.utils.typingx
StorageDict = dict[str, Union[cp.ndarray. np.ndarray]]

class Saturation(DiagnosticComponent):
    def __init__(
        self,
        computational_grid: ComputationalGrid,
        lphylin: bool,
        yoethf_parameters: Optional[dict[str, float]] = None,
        yomcst_parameters: Optional[dict[str, float]] = None,
        gt4py_config: GT4PyConfig,
    ) -> None:
        super().__init__(computational_grid, gt4py_config)
        externals = {"LPHYLIN": lphylin, "QMAX": 0.5}
        externals.update(yoethf_parameters or {})
        externals.update(yomcst_parameters or {})
        self.saturation = self.compile_stencil("saturation", externals)

    @cached_property
    def _input_properties(self):
        return {"ap": {"grid": (I, J, K), "units": "Pa"}, "t": {"grid": (I, J, K), "units": "K"}}

    @cached_property
    def _diagnostic_properties(self):
        return {"qsat": {"grid": (I, J, K), "units": "g g^-1"}}

    def array_call(self, state: StorageDict, out: StorageDict) -> None:
        self.saturation(
            in_ap=state["ap"],
            in_t=state["t"],
            out_qsat=out["qsat"],
            origin=(0, 0, 0),
            domain=self.computational_grid.grids[I, J, K].shape,
        )
            \end{minted}

            \caption{A Python class to compute the saturation water vapor pressure given the air pressure and temperature. Abridged excerpt from the CLOUDSC2-GT4Py dwarf.}
            \label{lst:saturation-infrastructure}
        \end{listing}

        Sympl components interact through dictionaries whose keys are the names of the model variables (fields), and whose values are xarray's \pyinline{DataArray}s \citep{hoyer17} collecting the grid point values, the labelled dimensions, the axis coordinates, and the units for those variables. The most relevant component exposed by Sympl is \pyinline{TendencyComponent}, producing tendencies for prognostic variables and retrieving diagnostics. The class defines a minimal interface to declare the list of input and output fields, and initialize and run an instance of the class. This imposes minor constraints on model developers when writing a new physics package.

        The bespoke infrastructure code for CLOUDSC-GT4Py and CLOUDSC2-GT4Py is bundled as an installable Python package called \pyinline{ifs_physics_common}. Not only does it build upon Sympl, but also extends it with grid-aware and stencil-oriented functionalities. Both the CLOUDSC cloud microphysics and the non-linear, tangent-linear and adjoint formulations of CLOUDSC2 are encoded as stand-alone \pyinline{TendencyComponent} classes settled over a \pyinline{ComputationalGrid}. The latter is a collection of index spaces for different grid locations. For instance, \pyinline{(I, J, K)} corresponds to cell centers, while \pyinline{(I, J, K-1/2)} denotes vertically-staggered grid points. For any input and output field, its name, units and grid location are specified as class properties. When running the component via the \emph{dunder} method \pyinline{__call__}, Sympl transparently extracts the raw data from the input \pyinline{DataArray}s according to the information provided in the class definition. This step may involve units conversion and axis transposition. The resulting storages are forwarded to the method \pyinline{array_call}, which carries out the actual computations, possibly by executing GT4Py stencil kernels.

        Listing \ref{lst:saturation-infrastructure} brings a concrete example from CLOUDSC2-GT4Py: a model component leveraging the stencil defined in Listing \ref{lst:saturation-stencil} to compute the saturation water vapor pressure. The class inherits \pyinline{DiagnosticComponent}, a stripped-down version of \pyinline{TendencyComponent}, which only retrieves diagnostic quantities. Within the instance initializer \pyinline{__init__}, the stencil from Listing \ref{lst:saturation-stencil}, registered using the decorator \pyinline{ifs_physics_common.framework.stencil.stencil_collection}, is compiled using the utility method \pyinline{compile_stencil}. The options configuring the stencil compilation (e.g.\,the GT4Py backend) are fetched from the dataclass \pyinline{GT4PyConfig}.

    %\biblio
\end{document}
